\documentclass[11pt]{article}
\usepackage{latexsym}
\usepackage{amsmath}
\usepackage{amssymb}
\usepackage{amsthm}
\usepackage{epsfig}
\usepackage{graphicx}
\usepackage{algorithm}
\usepackage{algorithmic}

\newcommand{\handout}[5]{
  \noindent
  \begin{center}
  \framebox{
    \vbox{
      \hbox to 5.78in { {\bf ICS 443: Parallel Algorithms} \hfill #2 }
      \vspace{4mm}
      \hbox to 5.78in { {\Large \hfill #5  \hfill} }
      \vspace{2mm}
      \hbox to 5.78in { {\em #3 \hfill #4} }
    }
  }
  \end{center}
  \vspace*{4mm}
}

\newcommand{\lecture}[4]{\handout{#1}{#2}{#3}{Scribe: #4}{Lecture #1}}

\newtheorem{theorem}{Theorem}
\newtheorem{corollary}[theorem]{Corollary}
\newtheorem{lemma}[theorem]{Lemma}
\newtheorem{observation}[theorem]{Observation}
\newtheorem{proposition}[theorem]{Proposition}
\newtheorem{definition}[theorem]{Definition}
\newtheorem{claim}[theorem]{Claim}
\newtheorem{fact}[theorem]{Fact}
\newtheorem{assumption}[theorem]{Assumption}

% 1-inch margins, from fullpage.sty by H.Partl, Version 2, Dec. 15, 1988.
\topmargin 0pt
\advance \topmargin by -\headheight
\advance \topmargin by -\headsep
\textheight 8.9in
\oddsidemargin 0pt
\evensidemargin \oddsidemargin
\marginparwidth 0.5in
\textwidth 6.5in

\parindent 0in
\parskip 1.5ex
%\renewcommand{\baselinestretch}{1.25}

\begin{document}

\lecture{12 --- October 2, 2017}{Fall 2017}{Prof.\ Nodari Sitchinava}{Bobby Figgs, Jipeng Huang}

\section{Overview}

In the last lecture we did a rough overview of how a work-efficient $O(\log n)$ time merging algorithm might work.

In this lecture we will review how the work-efficient algorithm works, and then show the pseudo code to implement this algorithm.

\section{Parallel Merging}

In this section, we will give an overview of how the parallel merging algorithm should work.
We start by stating our assumptions. We are assuming that all elements in both arrays are distinct. We are also assuming that the two arrays are sorted and of the same size. Let's now call our two input arrays $A$ and $B$. We will call the output array $C$. What we want this algorithm to do is have each processor grab a chunk of values from $A$ and merge it with the appropriate values from $B$.

The first step the algorithm takes, is to split array $A$ into sub arrays, each of size $k$. Now we want to grab the last element of each of these sub-arrays. We will label the \textbf{index} as $a_k, a_{k+1}, ... $. These values represent the bounds of our chunks of values from $A$, so as we can see, we have a chunk that we would like to merge that starts at $a_k$ and ends at $a_{k+1}$. But now we need to find which value in $B$ are appropriate to merge with this chunk.

The next step is to find the values in $B$ that we can safely merge with. To do this we need to use the rank function. The rank, $rank_Z(x)$, is defined as the index of the largest value in $Z$ that is less than or equal to $x$. This is useful for us in finding the bounds of the chunk of values in $B$. We can find the left bound of the $B$ chunk of values by plugging the value of left-bound of $A$ $(a_k)$ into the rank function. E.g. $rank_B(A[a_k])$. Similarly we can obtain the right-bound of the $B$ chunk with, $rank_B(A[a_{k+1}])$. The diagram below shows a visual representation of a single chunk in our arrays.

\begin{center}
\begin{tabular}{ |c|c|c|c|c|c|c| }
  \multicolumn{1}{c}{} &
  \multicolumn{1}{c}{} &
  \multicolumn{1}{c}{$a_k$} &
  \multicolumn{1}{c}{} &
  \multicolumn{1}{c}{$a_{k+1}$} &
  \multicolumn{1}{c}{} &
  \multicolumn{1}{c}{}\\
\hline
 $A$ & 1 & 3 & 5 & 7 & 9 & ... \\
 \hline
  \multicolumn{1}{c}{} &
  \multicolumn{1}{c}{} &
  \multicolumn{1}{c}{$\downarrow$} &
  \multicolumn{1}{c}{} &
  \multicolumn{1}{c}{$\downarrow$} &
  \multicolumn{1}{c}{} &
  \multicolumn{1}{c}{} \\
 \hline
 $B$ & 0 & 2 & 4 & 6 & 8 & ... \\
 \hline
 \multicolumn{1}{c}{} &
 \multicolumn{1}{c}{} &
 \multicolumn{1}{c}{$rank_B(A[a_k])$} &
 \multicolumn{1}{c}{} &
 \multicolumn{1}{c}{ $rank_B(A[a_{k+1}])$ }&
 \multicolumn{1}{c}{} &
 \multicolumn{1}{c}{} \\

\end{tabular}
\end{center}

Now that we have our chunks defined for both arrays, we can define the chunk in array $C$ lets call it $C[...]$. We can now say that $C[...]$ is equal to the union of our chunk from $A$ and our chunk from $B$. Formally,
$C[...] = A[a_k+1 ... a_{k+1}] \cup B[rank_B(A[a_k])+1 ... rank_B(A[a_{k+1}])]$. We have selected our chunk in such a way that, everything before our chunk in $A$ is less than every value in our chunk in $C$, and every value after our chunk in $A$ is greater than every value in $C$. Similarly everything before our chunk in $B$ is less than everything in $C$ and everything after our chunk in $B$ is greater than everything in $C$. Here I will prove that everything before the chunk in $A$ is less than everything in $C$

\subsection{Proof}
We want to prove that all elements in $A$ less than or equal to index $a_k$ is less than every element in $C$. Formally,  $\forall i \leq a_k : A[i] < C[...]$. We know that $C[...] = A[a_k+1 ... a_{k+1}] \cup B[rank_B(A[a_k])+1 ... rank_B(A[a_{k+1}])]$. Since $C[...]$ consists of elements of a chunk from $A$ and a chunk from $B$, we need to show that $\forall i \leq a_k : A[i] < A[a_k+1 ... a_{k+1}]$ AND $\forall i \leq a_k : A[i] < B[rank_B(A[a_k])+1 ... rank_B(A[a_{k+1}])]$

Let's start by proving, $\forall i \leq a_k : A[i] < A[a_k+1 ... a_{k+1}]$. This holds trivially because we know that array $A$ is sorted, and every element is distinct. Every element in a sorted array is less than the elements after it, and since our array starts at $a_k+1$ then $a_k$ must be less than it.

Now let's prove $\forall i \leq a_k : A[i] < B[rank_B(A[a_k])+1 ... rank_B(A[a_{k+1}])]$. We know that both of these arrays are sorted, so all we need to do is prove that all of our values are less than the first element. If our values are less than the first element than we know that they are less than the entire array. We also know that $a_k$ is the last element before our chunk in $A$, meaning that it is also the largest value in that area. Since array $A$ is sorted we just need to show that $A[a_k] < B[rank_B(A[a_k])+1]$. We know that the rank function will give us the \textbf{largest} value in $B$ that is less than or equal to $a_k$. But since every element is distinct it will actually be less than $A[a_k]$, i.e. $B[rank_B(A[a_k])+1] > A[a_k]$  And since we already know that $B$ is sorted, we know that $A[a_k]$ is less than every value in our chunk of $B$.

Since we have shown that $\forall i \leq a_k : A[i] < A[a_k+1 ... a_{k+1}]$ AND $\forall i \leq a_k : A[i] < B[rank_B(A[a_k])+1 ... rank_B(A[a_{k+1}])]$ is true. Then $\forall i \leq a_k : A[i] < C[...]$ is also true.

\subsection{Algorithm}
Now, we can start to write the algorithm for the parallel merging sort using above what we prove.
Before writing algorithm, we should analyze how it works.  First, we chunks the array A into a smaller array and each array carries $\log n$ elements.  For the each last element of array $A_1, A_2, A_3 ... A_n$,  we find it rank in array B so we get $B_1, B_2, B_3 ... B_n$.  IF the size of array $B_k$ is greater than $\log n$ , it jumps to recursive function and switchs $A_k$ with $B_k$.  However, the recursive function only runs once since after switching the size of array $B_k$ will be smaller than $\log n$. Therefore, we can parallel do the merge sort for each $A_1 \cup B_1$ .... and put in array $C$ in order.  The $C$ array is the final answer.

\begin{center}
\begin{tabular}{ |c|c|c|c|c|c|c| }
  \multicolumn{1}{c}{} &
  \multicolumn{1}{c}{$A_1$} &
  \multicolumn{1}{c}{$A_2$} &
  \multicolumn{1}{c}{$A_3$} &
  \multicolumn{1}{c}{$A_4$} &
  \multicolumn{1}{c}{}\\
\hline
 $A$  & & & & & & \\
 \hline
  \multicolumn{1}{c}{} &
  \multicolumn{1}{c}{$\downarrow$} &
  \multicolumn{1}{c}{$\downarrow$} &
  \multicolumn{1}{c}{$\downarrow$} &
  \multicolumn{1}{c}{$\downarrow$} &
  \multicolumn{1}{c}{} \\
 \hline
 $B$ & & & & & & \\
 \hline
 \multicolumn{1}{c}{} &
 \multicolumn{1}{c}{$B_1$} &
 \multicolumn{1}{c}{$B_2$} &
 \multicolumn{1}{c}{$B_3$} &
 \multicolumn{1}{c}{$B_4$} &
 \multicolumn{1}{c}{} \\
 \hline
 $C$  & $A_1 \cup B_1$ &$A_2 \cup B_2$  &$A_3 \cup B_3$  & $A_4 \cup B_4$ & & \\
 \hline
 \multicolumn{1}{c}{} &
 \multicolumn{1}{c}{$A_1$} &
 \multicolumn{1}{c}{$A_2$} &
 \multicolumn{1}{c}{$A_3$} &
 \multicolumn{1}{c}{$A_4$} &
 \multicolumn{1}{c}{}\\

\end{tabular}
\end{center}


\begin{algorithm}
\caption{Parallel-Merge}
\begin{algorithmic}
\STATE $n' = \lfloor n \log n \rfloor$
\FOR{$k = 1~to~n' + 1 {\bf~in~parallel}$}
  \STATE $r[k] = rank( A[k - \log n] , B)$
  \IF {$(r_k+1 - r[k]+1) \leq \log n$}
  \STATE $C[(k\log n + r[k] + 1) ... (k+1)\log n + r[k+1]] = $seq-merge$(A[k-\log n +1...(k+1)\log n)], B[r[k] + 1 ... r[k + 1]])$
  \ELSE
  \STATE $C[(k\log n + r[k] + 1) ... (k+1)\log n + r[k+1]] = $Parallel-Merge$(B[r[k] + 1 ... r[k + 1]] , A[k-\log n +1...(k+1)\log n)])$
  \ENDIF
\ENDFOR
\end{algorithmic}
\end{algorithm}

\section{Conclusion}
For this lecture, we find a more work-efficient in $O(\log n)$ time merging sort in parallel and also partially prove that.  At the next lecture we will completely prove it works step by step.

%\bibliography{mybib}
\bibliographystyle{alpha}

\end{document}
